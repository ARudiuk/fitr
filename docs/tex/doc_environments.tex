\section{\texorpdfstring{\texttt{fitr.environments}}{fitr.environments}}\label{fitr.environments}

Functions to synthesize data from behavioural tasks.

\subsection{Graph}\label{graph}

\begin{Shaded}
\begin{Highlighting}[]
\NormalTok{fitr.environments.Graph()}
\end{Highlighting}
\end{Shaded}

Base object that defines a reinforcement learning task.

\paragraph{Definitions}\label{definitions}

\begin{itemize}
\tightlist
\item
  \(\mathbf x \in \mathcal X\) be a one-hot state vector, where
  \(|\mathcal X|=n_x\)
\item
  \(\mathbf u \in \mathcal U\) be a one-hot action vector, where
  \(|\mathcal U|=n_u\)
\item
  \(\mathsf T = p(\mathbf x_{t+1}|\mathbf x_t, \mathbf u_t)\) be a
  transition tensor
\item
  \(p(\mathbf x)\) be a distribution over starting states
\item
  \(\mathcal J: \mathcal X \to \mathcal R\), where
  \(\mathcal R \subseteq \mathbb R\) be a reward function
\end{itemize}

Arguments:

\begin{itemize}
\tightlist
\item
  \textbf{T}: Transition tensor
\item
  \textbf{R}: Vector of rewards for each state such that scalar reward
  \(r_t = \mathbf r^ op \mathbf x\)
\item
  \textbf{end\_states}: A vector \(\{0, 1\}^{n_x}\) identifying which
  states terminate a trial (aka episode)
\item
  \textbf{p\_start}: Initial state distribution
\item
  \textbf{label}: A string identifying a name for the task
\item
  \textbf{state\_labels}: A list or array of strings labeling the
  different states (for plotting purposes)
\item
  \textbf{action\_labels}: A list or array of strings labeling the
  different actions (for plotting purposes)
\item
  \textbf{rng}: \texttt{np.random.RandomState} object
\item
  \textbf{f\_reward}: A function whose first argument is a vector of
  rewards for each state, and whose second argument is a state vector,
  and whose output is a scalar reward
\item
  \textbf{cmap}: Matplotlib colormap for plotting.
\end{itemize}

\paragraph{Notes}\label{notes}

There are two critical methods for the \texttt{Graph} class:
\texttt{observation()} and \texttt{step}. All instances of a
\texttt{Graph} must be able to call these functions. Let's say you have
some bandit task \texttt{MyBanditTask} that inherits from
\texttt{Graph}. To run such a task would look something like this:

\begin{Shaded}
\begin{Highlighting}[]
\NormalTok{env }\OperatorTok{=}\NormalTok{ MyBanditTask()            }\CommentTok{# Instantiate your environment object}
\NormalTok{agent }\OperatorTok{=}\NormalTok{ MyAgent()               }\CommentTok{# Some agent object (arbitrary, really)}
\ControlFlowTok{for}\NormalTok{ t }\KeywordTok{in} \BuiltInTok{range}\NormalTok{(ntrials):}
\NormalTok{    x }\OperatorTok{=}\NormalTok{ env.observation()       }\CommentTok{# Samples initial state}
\NormalTok{    u }\OperatorTok{=}\NormalTok{ agent.action(x)         }\CommentTok{# Choose some action}
\NormalTok{    x_, r, done }\OperatorTok{=}\NormalTok{ agent.step(u) }\CommentTok{# Transition based on action}
\end{Highlighting}
\end{Shaded}

What differentiates tasks are the transition tensor \(\mathsf T\),
starting state distribution \(p(\mathbf x)\) and reward function
\(\mathcal J\) (which here would include the reward vector
\(\mathbf r\)).

\begin{center}\rule{0.5\linewidth}{\linethickness}\end{center}

\subsubsection{Graph.adjacency\_matrix\_decomposition}\label{graph.adjacency_matrix_decomposition}

\begin{Shaded}
\begin{Highlighting}[]
\NormalTok{fitr.environments.adjacency_matrix_decomposition(}\VariableTok{self}\NormalTok{)}
\end{Highlighting}
\end{Shaded}

Singular value decomposition of the graph adjacency matrix

\begin{center}\rule{0.5\linewidth}{\linethickness}\end{center}

\subsubsection{Graph.get\_graph\_depth}\label{graph.get_graph_depth}

\begin{Shaded}
\begin{Highlighting}[]
\NormalTok{fitr.environments.get_graph_depth(}\VariableTok{self}\NormalTok{)}
\end{Highlighting}
\end{Shaded}

Returns the depth of the task graph.

Calculated as the depth from \texttt{START} (pre-initial state) to
\texttt{END} (which absorbs trial from all terminal states), minus 2 to
account for the \texttt{START-\textgreater{}node} \&
\texttt{node-\textgreater{}END} transitions.

Returns:

An \texttt{int} identifying the depth of the current graph for a single
trial of the task

\begin{center}\rule{0.5\linewidth}{\linethickness}\end{center}

\subsubsection{Graph.laplacian\_matrix\_decomposition}\label{graph.laplacian_matrix_decomposition}

\begin{Shaded}
\begin{Highlighting}[]
\NormalTok{fitr.environments.laplacian_matrix_decomposition(}\VariableTok{self}\NormalTok{)}
\end{Highlighting}
\end{Shaded}

Singular value decomposition of the graph Laplacian

\begin{center}\rule{0.5\linewidth}{\linethickness}\end{center}

\subsubsection{Graph.make\_action\_labels}\label{graph.make_action_labels}

\begin{Shaded}
\begin{Highlighting}[]
\NormalTok{fitr.environments.make_action_labels(}\VariableTok{self}\NormalTok{)}
\end{Highlighting}
\end{Shaded}

Creates labels for the actions (for plotting) if none provided

\begin{center}\rule{0.5\linewidth}{\linethickness}\end{center}

\subsubsection{Graph.make\_digraph}\label{graph.make_digraph}

\begin{Shaded}
\begin{Highlighting}[]
\NormalTok{fitr.environments.make_digraph(}\VariableTok{self}\NormalTok{)}
\end{Highlighting}
\end{Shaded}

Creates a \texttt{networkx} \texttt{DiGraph} object from the transition
tensor for the purpose of plotting and some other analyses.

\begin{center}\rule{0.5\linewidth}{\linethickness}\end{center}

\subsubsection{Graph.make\_state\_labels}\label{graph.make_state_labels}

\begin{Shaded}
\begin{Highlighting}[]
\NormalTok{fitr.environments.make_state_labels(}\VariableTok{self}\NormalTok{)}
\end{Highlighting}
\end{Shaded}

Creates labels for the states (for plotting) if none provided

\begin{center}\rule{0.5\linewidth}{\linethickness}\end{center}

\subsubsection{Graph.make\_undirected\_graph}\label{graph.make_undirected_graph}

\begin{Shaded}
\begin{Highlighting}[]
\NormalTok{fitr.environments.make_undirected_graph(}\VariableTok{self}\NormalTok{)}
\end{Highlighting}
\end{Shaded}

Converts the DiGraph to undirected and computes some stats

\begin{center}\rule{0.5\linewidth}{\linethickness}\end{center}

\subsubsection{Graph.observation}\label{graph.observation}

\begin{Shaded}
\begin{Highlighting}[]
\NormalTok{fitr.environments.observation(}\VariableTok{self}\NormalTok{)}
\end{Highlighting}
\end{Shaded}

Samples an initial state from the start-state distribution
\(p(\mathbf x)\)

\[
\mathbf x_0 \sim p(\mathbf x)
\]

Returns:

A one-hot vector \texttt{ndarray((nstates,))} indicating the starting
state.

Examples:

\begin{Shaded}
\begin{Highlighting}[]
\NormalTok{x }\OperatorTok{=}\NormalTok{ env.observation()}
\end{Highlighting}
\end{Shaded}

\begin{center}\rule{0.5\linewidth}{\linethickness}\end{center}

\subsubsection{Graph.plot\_action\_outcome\_probabilities}\label{graph.plot_action_outcome_probabilities}

\begin{Shaded}
\begin{Highlighting}[]
\NormalTok{fitr.environments.plot_action_outcome_probabilities(}\VariableTok{self}\NormalTok{, figsize}\OperatorTok{=}\VariableTok{None}\NormalTok{, outfile}\OperatorTok{=}\VariableTok{None}\NormalTok{, outfiletype}\OperatorTok{=}\StringTok{'pdf'}\NormalTok{, cmap}\OperatorTok{=}\StringTok{'Greys_r'}\NormalTok{)}
\end{Highlighting}
\end{Shaded}

Plots the probabilities of different outcomes given actions.

Each plot is a heatmap for a starting state showing the transition
probabilities for each action-outcome pair within that state.

\begin{center}\rule{0.5\linewidth}{\linethickness}\end{center}

\subsubsection{Graph.plot\_graph}\label{graph.plot_graph}

\begin{Shaded}
\begin{Highlighting}[]
\NormalTok{fitr.environments.plot_graph(}\VariableTok{self}\NormalTok{, figsize}\OperatorTok{=}\VariableTok{None}\NormalTok{, node_size}\OperatorTok{=}\DecValTok{2000}\NormalTok{, arrowsize}\OperatorTok{=}\DecValTok{20}\NormalTok{, lw}\OperatorTok{=}\FloatTok{1.5}\NormalTok{, font_size}\OperatorTok{=}\DecValTok{12}\NormalTok{, title}\OperatorTok{=}\VariableTok{False}\NormalTok{, outfile}\OperatorTok{=}\VariableTok{None}\NormalTok{, outfiletype}\OperatorTok{=}\StringTok{'pdf'}\NormalTok{)}
\end{Highlighting}
\end{Shaded}

Plots the directed graph of the task

\begin{center}\rule{0.5\linewidth}{\linethickness}\end{center}

\subsubsection{Graph.plot\_spectral\_properties}\label{graph.plot_spectral_properties}

\begin{Shaded}
\begin{Highlighting}[]
\NormalTok{fitr.environments.plot_spectral_properties(}\VariableTok{self}\NormalTok{, figsize}\OperatorTok{=}\VariableTok{None}\NormalTok{, outfile}\OperatorTok{=}\VariableTok{None}\NormalTok{, outfiletype}\OperatorTok{=}\StringTok{'pdf'}\NormalTok{)}
\end{Highlighting}
\end{Shaded}

Creates a set of subplots depicting the graph Laplacian and its spectral
decomposition.

\begin{center}\rule{0.5\linewidth}{\linethickness}\end{center}

\subsubsection{Graph.random\_action}\label{graph.random_action}

\begin{Shaded}
\begin{Highlighting}[]
\NormalTok{fitr.environments.random_action(}\VariableTok{self}\NormalTok{)}
\end{Highlighting}
\end{Shaded}

Samples a random one-hot action vector uniformly over the action space.

Useful for testing that your environment works, without having to create
an agent.

\[
\mathbf u \sim \mathrm{Multinomial}\Big(1, \mathbf p=\{p_i = \frac{1}{|\mathcal U|}\}_{i=1}^{|\mathcal U|}\Big)
\]

Returns:

A one-hot action vector of type \texttt{ndarray((nactions,))}

Examples:

\begin{Shaded}
\begin{Highlighting}[]
\NormalTok{u }\OperatorTok{=}\NormalTok{ env.random_action()}
\end{Highlighting}
\end{Shaded}

\begin{center}\rule{0.5\linewidth}{\linethickness}\end{center}

\subsubsection{Graph.step}\label{graph.step}

\begin{Shaded}
\begin{Highlighting}[]
\NormalTok{fitr.environments.step(}\VariableTok{self}\NormalTok{, action)}
\end{Highlighting}
\end{Shaded}

Executes a state transition in the environment.

Arguments:

action : A one-hot vector of type \texttt{ndarray((naction,))}
indicating the action selected at the current state.

Returns:

A 3-tuple representing the next state (\texttt{ndarray((noutcomes,))}),
scalar reward, and whether the current step terminates a trial
(\texttt{bool}).

Raises:

\texttt{RuntimeError} if \texttt{env.observation()} not called after a
previous \texttt{env.step(...)} call yielded a terminal state.

\begin{center}\rule{0.5\linewidth}{\linethickness}\end{center}

\subsection{TwoArmedBandit}\label{twoarmedbandit}

\begin{Shaded}
\begin{Highlighting}[]
\NormalTok{fitr.environments.TwoArmedBandit()}
\end{Highlighting}
\end{Shaded}

A simple 2-armed bandit task

\begin{center}\rule{0.5\linewidth}{\linethickness}\end{center}

\subsubsection{TwoArmedBandit.adjacency\_matrix\_decomposition}\label{twoarmedbandit.adjacency_matrix_decomposition}

\begin{Shaded}
\begin{Highlighting}[]
\NormalTok{fitr.environments.adjacency_matrix_decomposition(}\VariableTok{self}\NormalTok{)}
\end{Highlighting}
\end{Shaded}

Singular value decomposition of the graph adjacency matrix

\begin{center}\rule{0.5\linewidth}{\linethickness}\end{center}

\subsubsection{TwoArmedBandit.get\_graph\_depth}\label{twoarmedbandit.get_graph_depth}

\begin{Shaded}
\begin{Highlighting}[]
\NormalTok{fitr.environments.get_graph_depth(}\VariableTok{self}\NormalTok{)}
\end{Highlighting}
\end{Shaded}

Returns the depth of the task graph.

Calculated as the depth from \texttt{START} (pre-initial state) to
\texttt{END} (which absorbs trial from all terminal states), minus 2 to
account for the \texttt{START-\textgreater{}node} \&
\texttt{node-\textgreater{}END} transitions.

Returns:

An \texttt{int} identifying the depth of the current graph for a single
trial of the task

\begin{center}\rule{0.5\linewidth}{\linethickness}\end{center}

\subsubsection{TwoArmedBandit.laplacian\_matrix\_decomposition}\label{twoarmedbandit.laplacian_matrix_decomposition}

\begin{Shaded}
\begin{Highlighting}[]
\NormalTok{fitr.environments.laplacian_matrix_decomposition(}\VariableTok{self}\NormalTok{)}
\end{Highlighting}
\end{Shaded}

Singular value decomposition of the graph Laplacian

\begin{center}\rule{0.5\linewidth}{\linethickness}\end{center}

\subsubsection{TwoArmedBandit.make\_action\_labels}\label{twoarmedbandit.make_action_labels}

\begin{Shaded}
\begin{Highlighting}[]
\NormalTok{fitr.environments.make_action_labels(}\VariableTok{self}\NormalTok{)}
\end{Highlighting}
\end{Shaded}

Creates labels for the actions (for plotting) if none provided

\begin{center}\rule{0.5\linewidth}{\linethickness}\end{center}

\subsubsection{TwoArmedBandit.make\_digraph}\label{twoarmedbandit.make_digraph}

\begin{Shaded}
\begin{Highlighting}[]
\NormalTok{fitr.environments.make_digraph(}\VariableTok{self}\NormalTok{)}
\end{Highlighting}
\end{Shaded}

Creates a \texttt{networkx} \texttt{DiGraph} object from the transition
tensor for the purpose of plotting and some other analyses.

\begin{center}\rule{0.5\linewidth}{\linethickness}\end{center}

\subsubsection{TwoArmedBandit.make\_state\_labels}\label{twoarmedbandit.make_state_labels}

\begin{Shaded}
\begin{Highlighting}[]
\NormalTok{fitr.environments.make_state_labels(}\VariableTok{self}\NormalTok{)}
\end{Highlighting}
\end{Shaded}

Creates labels for the states (for plotting) if none provided

\begin{center}\rule{0.5\linewidth}{\linethickness}\end{center}

\subsubsection{TwoArmedBandit.make\_undirected\_graph}\label{twoarmedbandit.make_undirected_graph}

\begin{Shaded}
\begin{Highlighting}[]
\NormalTok{fitr.environments.make_undirected_graph(}\VariableTok{self}\NormalTok{)}
\end{Highlighting}
\end{Shaded}

Converts the DiGraph to undirected and computes some stats

\begin{center}\rule{0.5\linewidth}{\linethickness}\end{center}

\subsubsection{TwoArmedBandit.observation}\label{twoarmedbandit.observation}

\begin{Shaded}
\begin{Highlighting}[]
\NormalTok{fitr.environments.observation(}\VariableTok{self}\NormalTok{)}
\end{Highlighting}
\end{Shaded}

Samples an initial state from the start-state distribution
\(p(\mathbf x)\)

\[
\mathbf x_0 \sim p(\mathbf x)
\]

Returns:

A one-hot vector \texttt{ndarray((nstates,))} indicating the starting
state.

Examples:

\begin{Shaded}
\begin{Highlighting}[]
\NormalTok{x }\OperatorTok{=}\NormalTok{ env.observation()}
\end{Highlighting}
\end{Shaded}

\begin{center}\rule{0.5\linewidth}{\linethickness}\end{center}

\subsubsection{TwoArmedBandit.plot\_action\_outcome\_probabilities}\label{twoarmedbandit.plot_action_outcome_probabilities}

\begin{Shaded}
\begin{Highlighting}[]
\NormalTok{fitr.environments.plot_action_outcome_probabilities(}\VariableTok{self}\NormalTok{, figsize}\OperatorTok{=}\VariableTok{None}\NormalTok{, outfile}\OperatorTok{=}\VariableTok{None}\NormalTok{, outfiletype}\OperatorTok{=}\StringTok{'pdf'}\NormalTok{, cmap}\OperatorTok{=}\StringTok{'Greys_r'}\NormalTok{)}
\end{Highlighting}
\end{Shaded}

Plots the probabilities of different outcomes given actions.

Each plot is a heatmap for a starting state showing the transition
probabilities for each action-outcome pair within that state.

\begin{center}\rule{0.5\linewidth}{\linethickness}\end{center}

\subsubsection{TwoArmedBandit.plot\_graph}\label{twoarmedbandit.plot_graph}

\begin{Shaded}
\begin{Highlighting}[]
\NormalTok{fitr.environments.plot_graph(}\VariableTok{self}\NormalTok{, figsize}\OperatorTok{=}\VariableTok{None}\NormalTok{, node_size}\OperatorTok{=}\DecValTok{2000}\NormalTok{, arrowsize}\OperatorTok{=}\DecValTok{20}\NormalTok{, lw}\OperatorTok{=}\FloatTok{1.5}\NormalTok{, font_size}\OperatorTok{=}\DecValTok{12}\NormalTok{, title}\OperatorTok{=}\VariableTok{False}\NormalTok{, outfile}\OperatorTok{=}\VariableTok{None}\NormalTok{, outfiletype}\OperatorTok{=}\StringTok{'pdf'}\NormalTok{)}
\end{Highlighting}
\end{Shaded}

Plots the directed graph of the task

\begin{center}\rule{0.5\linewidth}{\linethickness}\end{center}

\subsubsection{TwoArmedBandit.plot\_spectral\_properties}\label{twoarmedbandit.plot_spectral_properties}

\begin{Shaded}
\begin{Highlighting}[]
\NormalTok{fitr.environments.plot_spectral_properties(}\VariableTok{self}\NormalTok{, figsize}\OperatorTok{=}\VariableTok{None}\NormalTok{, outfile}\OperatorTok{=}\VariableTok{None}\NormalTok{, outfiletype}\OperatorTok{=}\StringTok{'pdf'}\NormalTok{)}
\end{Highlighting}
\end{Shaded}

Creates a set of subplots depicting the graph Laplacian and its spectral
decomposition.

\begin{center}\rule{0.5\linewidth}{\linethickness}\end{center}

\subsubsection{TwoArmedBandit.random\_action}\label{twoarmedbandit.random_action}

\begin{Shaded}
\begin{Highlighting}[]
\NormalTok{fitr.environments.random_action(}\VariableTok{self}\NormalTok{)}
\end{Highlighting}
\end{Shaded}

Samples a random one-hot action vector uniformly over the action space.

Useful for testing that your environment works, without having to create
an agent.

\[
\mathbf u \sim \mathrm{Multinomial}\Big(1, \mathbf p=\{p_i = \frac{1}{|\mathcal U|}\}_{i=1}^{|\mathcal U|}\Big)
\]

Returns:

A one-hot action vector of type \texttt{ndarray((nactions,))}

Examples:

\begin{Shaded}
\begin{Highlighting}[]
\NormalTok{u }\OperatorTok{=}\NormalTok{ env.random_action()}
\end{Highlighting}
\end{Shaded}

\begin{center}\rule{0.5\linewidth}{\linethickness}\end{center}

\subsubsection{TwoArmedBandit.step}\label{twoarmedbandit.step}

\begin{Shaded}
\begin{Highlighting}[]
\NormalTok{fitr.environments.step(}\VariableTok{self}\NormalTok{, action)}
\end{Highlighting}
\end{Shaded}

Executes a state transition in the environment.

Arguments:

action : A one-hot vector of type \texttt{ndarray((naction,))}
indicating the action selected at the current state.

Returns:

A 3-tuple representing the next state (\texttt{ndarray((noutcomes,))}),
scalar reward, and whether the current step terminates a trial
(\texttt{bool}).

Raises:

\texttt{RuntimeError} if \texttt{env.observation()} not called after a
previous \texttt{env.step(...)} call yielded a terminal state.

\begin{center}\rule{0.5\linewidth}{\linethickness}\end{center}

\subsection{OrthogonalGoNoGo}\label{orthogonalgonogo}

\begin{Shaded}
\begin{Highlighting}[]
\NormalTok{fitr.environments.OrthogonalGoNoGo()}
\end{Highlighting}
\end{Shaded}

The Orthogonal GoNogo task from Guitart-Masip et al. (2012)

\begin{center}\rule{0.5\linewidth}{\linethickness}\end{center}

\subsubsection{OrthogonalGoNoGo.adjacency\_matrix\_decomposition}\label{orthogonalgonogo.adjacency_matrix_decomposition}

\begin{Shaded}
\begin{Highlighting}[]
\NormalTok{fitr.environments.adjacency_matrix_decomposition(}\VariableTok{self}\NormalTok{)}
\end{Highlighting}
\end{Shaded}

Singular value decomposition of the graph adjacency matrix

\begin{center}\rule{0.5\linewidth}{\linethickness}\end{center}

\subsubsection{OrthogonalGoNoGo.get\_graph\_depth}\label{orthogonalgonogo.get_graph_depth}

\begin{Shaded}
\begin{Highlighting}[]
\NormalTok{fitr.environments.get_graph_depth(}\VariableTok{self}\NormalTok{)}
\end{Highlighting}
\end{Shaded}

Returns the depth of the task graph.

Calculated as the depth from \texttt{START} (pre-initial state) to
\texttt{END} (which absorbs trial from all terminal states), minus 2 to
account for the \texttt{START-\textgreater{}node} \&
\texttt{node-\textgreater{}END} transitions.

Returns:

An \texttt{int} identifying the depth of the current graph for a single
trial of the task

\begin{center}\rule{0.5\linewidth}{\linethickness}\end{center}

\subsubsection{OrthogonalGoNoGo.laplacian\_matrix\_decomposition}\label{orthogonalgonogo.laplacian_matrix_decomposition}

\begin{Shaded}
\begin{Highlighting}[]
\NormalTok{fitr.environments.laplacian_matrix_decomposition(}\VariableTok{self}\NormalTok{)}
\end{Highlighting}
\end{Shaded}

Singular value decomposition of the graph Laplacian

\begin{center}\rule{0.5\linewidth}{\linethickness}\end{center}

\subsubsection{OrthogonalGoNoGo.make\_action\_labels}\label{orthogonalgonogo.make_action_labels}

\begin{Shaded}
\begin{Highlighting}[]
\NormalTok{fitr.environments.make_action_labels(}\VariableTok{self}\NormalTok{)}
\end{Highlighting}
\end{Shaded}

Creates labels for the actions (for plotting) if none provided

\begin{center}\rule{0.5\linewidth}{\linethickness}\end{center}

\subsubsection{OrthogonalGoNoGo.make\_digraph}\label{orthogonalgonogo.make_digraph}

\begin{Shaded}
\begin{Highlighting}[]
\NormalTok{fitr.environments.make_digraph(}\VariableTok{self}\NormalTok{)}
\end{Highlighting}
\end{Shaded}

Creates a \texttt{networkx} \texttt{DiGraph} object from the transition
tensor for the purpose of plotting and some other analyses.

\begin{center}\rule{0.5\linewidth}{\linethickness}\end{center}

\subsubsection{OrthogonalGoNoGo.make\_state\_labels}\label{orthogonalgonogo.make_state_labels}

\begin{Shaded}
\begin{Highlighting}[]
\NormalTok{fitr.environments.make_state_labels(}\VariableTok{self}\NormalTok{)}
\end{Highlighting}
\end{Shaded}

Creates labels for the states (for plotting) if none provided

\begin{center}\rule{0.5\linewidth}{\linethickness}\end{center}

\subsubsection{OrthogonalGoNoGo.make\_undirected\_graph}\label{orthogonalgonogo.make_undirected_graph}

\begin{Shaded}
\begin{Highlighting}[]
\NormalTok{fitr.environments.make_undirected_graph(}\VariableTok{self}\NormalTok{)}
\end{Highlighting}
\end{Shaded}

Converts the DiGraph to undirected and computes some stats

\begin{center}\rule{0.5\linewidth}{\linethickness}\end{center}

\subsubsection{OrthogonalGoNoGo.observation}\label{orthogonalgonogo.observation}

\begin{Shaded}
\begin{Highlighting}[]
\NormalTok{fitr.environments.observation(}\VariableTok{self}\NormalTok{)}
\end{Highlighting}
\end{Shaded}

Samples an initial state from the start-state distribution
\(p(\mathbf x)\)

\[
\mathbf x_0 \sim p(\mathbf x)
\]

Returns:

A one-hot vector \texttt{ndarray((nstates,))} indicating the starting
state.

Examples:

\begin{Shaded}
\begin{Highlighting}[]
\NormalTok{x }\OperatorTok{=}\NormalTok{ env.observation()}
\end{Highlighting}
\end{Shaded}

\begin{center}\rule{0.5\linewidth}{\linethickness}\end{center}

\subsubsection{OrthogonalGoNoGo.plot\_action\_outcome\_probabilities}\label{orthogonalgonogo.plot_action_outcome_probabilities}

\begin{Shaded}
\begin{Highlighting}[]
\NormalTok{fitr.environments.plot_action_outcome_probabilities(}\VariableTok{self}\NormalTok{, figsize}\OperatorTok{=}\VariableTok{None}\NormalTok{, outfile}\OperatorTok{=}\VariableTok{None}\NormalTok{, outfiletype}\OperatorTok{=}\StringTok{'pdf'}\NormalTok{, cmap}\OperatorTok{=}\StringTok{'Greys_r'}\NormalTok{)}
\end{Highlighting}
\end{Shaded}

Plots the probabilities of different outcomes given actions.

Each plot is a heatmap for a starting state showing the transition
probabilities for each action-outcome pair within that state.

\begin{center}\rule{0.5\linewidth}{\linethickness}\end{center}

\subsubsection{OrthogonalGoNoGo.plot\_graph}\label{orthogonalgonogo.plot_graph}

\begin{Shaded}
\begin{Highlighting}[]
\NormalTok{fitr.environments.plot_graph(}\VariableTok{self}\NormalTok{, figsize}\OperatorTok{=}\VariableTok{None}\NormalTok{, node_size}\OperatorTok{=}\DecValTok{2000}\NormalTok{, arrowsize}\OperatorTok{=}\DecValTok{20}\NormalTok{, lw}\OperatorTok{=}\FloatTok{1.5}\NormalTok{, font_size}\OperatorTok{=}\DecValTok{12}\NormalTok{, title}\OperatorTok{=}\VariableTok{False}\NormalTok{, outfile}\OperatorTok{=}\VariableTok{None}\NormalTok{, outfiletype}\OperatorTok{=}\StringTok{'pdf'}\NormalTok{)}
\end{Highlighting}
\end{Shaded}

Plots the directed graph of the task

\begin{center}\rule{0.5\linewidth}{\linethickness}\end{center}

\subsubsection{OrthogonalGoNoGo.plot\_spectral\_properties}\label{orthogonalgonogo.plot_spectral_properties}

\begin{Shaded}
\begin{Highlighting}[]
\NormalTok{fitr.environments.plot_spectral_properties(}\VariableTok{self}\NormalTok{, figsize}\OperatorTok{=}\VariableTok{None}\NormalTok{, outfile}\OperatorTok{=}\VariableTok{None}\NormalTok{, outfiletype}\OperatorTok{=}\StringTok{'pdf'}\NormalTok{)}
\end{Highlighting}
\end{Shaded}

Creates a set of subplots depicting the graph Laplacian and its spectral
decomposition.

\begin{center}\rule{0.5\linewidth}{\linethickness}\end{center}

\subsubsection{OrthogonalGoNoGo.random\_action}\label{orthogonalgonogo.random_action}

\begin{Shaded}
\begin{Highlighting}[]
\NormalTok{fitr.environments.random_action(}\VariableTok{self}\NormalTok{)}
\end{Highlighting}
\end{Shaded}

Samples a random one-hot action vector uniformly over the action space.

Useful for testing that your environment works, without having to create
an agent.

\[
\mathbf u \sim \mathrm{Multinomial}\Big(1, \mathbf p=\{p_i = \frac{1}{|\mathcal U|}\}_{i=1}^{|\mathcal U|}\Big)
\]

Returns:

A one-hot action vector of type \texttt{ndarray((nactions,))}

Examples:

\begin{Shaded}
\begin{Highlighting}[]
\NormalTok{u }\OperatorTok{=}\NormalTok{ env.random_action()}
\end{Highlighting}
\end{Shaded}

\begin{center}\rule{0.5\linewidth}{\linethickness}\end{center}

\subsubsection{OrthogonalGoNoGo.step}\label{orthogonalgonogo.step}

\begin{Shaded}
\begin{Highlighting}[]
\NormalTok{fitr.environments.step(}\VariableTok{self}\NormalTok{, action)}
\end{Highlighting}
\end{Shaded}

Executes a state transition in the environment.

Arguments:

action : A one-hot vector of type \texttt{ndarray((naction,))}
indicating the action selected at the current state.

Returns:

A 3-tuple representing the next state (\texttt{ndarray((noutcomes,))}),
scalar reward, and whether the current step terminates a trial
(\texttt{bool}).

Raises:

\texttt{RuntimeError} if \texttt{env.observation()} not called after a
previous \texttt{env.step(...)} call yielded a terminal state.

\begin{center}\rule{0.5\linewidth}{\linethickness}\end{center}

\subsection{TwoStep}\label{twostep}

\begin{Shaded}
\begin{Highlighting}[]
\NormalTok{fitr.environments.TwoStep()}
\end{Highlighting}
\end{Shaded}

An implementation of the Two-Step Task from Daw et al. (2011).

Arguments:

\begin{itemize}
\tightlist
\item
  \textbf{mu}: \texttt{float} identifying the drift of the
  reward-determining Gaussian random walks
\item
  \textbf{sd}: \texttt{float} identifying the standard deviation of the
  reward-determining Gaussian random walks
\end{itemize}

\begin{center}\rule{0.5\linewidth}{\linethickness}\end{center}

\subsubsection{TwoStep.adjacency\_matrix\_decomposition}\label{twostep.adjacency_matrix_decomposition}

\begin{Shaded}
\begin{Highlighting}[]
\NormalTok{fitr.environments.adjacency_matrix_decomposition(}\VariableTok{self}\NormalTok{)}
\end{Highlighting}
\end{Shaded}

Singular value decomposition of the graph adjacency matrix

\begin{center}\rule{0.5\linewidth}{\linethickness}\end{center}

\subsubsection{TwoStep.f\_reward}\label{twostep.f_reward}

\begin{Shaded}
\begin{Highlighting}[]
\NormalTok{fitr.environments.f_reward(}\VariableTok{self}\NormalTok{, R, x)}
\end{Highlighting}
\end{Shaded}

\begin{center}\rule{0.5\linewidth}{\linethickness}\end{center}

\subsubsection{TwoStep.get\_graph\_depth}\label{twostep.get_graph_depth}

\begin{Shaded}
\begin{Highlighting}[]
\NormalTok{fitr.environments.get_graph_depth(}\VariableTok{self}\NormalTok{)}
\end{Highlighting}
\end{Shaded}

Returns the depth of the task graph.

Calculated as the depth from \texttt{START} (pre-initial state) to
\texttt{END} (which absorbs trial from all terminal states), minus 2 to
account for the \texttt{START-\textgreater{}node} \&
\texttt{node-\textgreater{}END} transitions.

Returns:

An \texttt{int} identifying the depth of the current graph for a single
trial of the task

\begin{center}\rule{0.5\linewidth}{\linethickness}\end{center}

\subsubsection{TwoStep.laplacian\_matrix\_decomposition}\label{twostep.laplacian_matrix_decomposition}

\begin{Shaded}
\begin{Highlighting}[]
\NormalTok{fitr.environments.laplacian_matrix_decomposition(}\VariableTok{self}\NormalTok{)}
\end{Highlighting}
\end{Shaded}

Singular value decomposition of the graph Laplacian

\begin{center}\rule{0.5\linewidth}{\linethickness}\end{center}

\subsubsection{TwoStep.make\_action\_labels}\label{twostep.make_action_labels}

\begin{Shaded}
\begin{Highlighting}[]
\NormalTok{fitr.environments.make_action_labels(}\VariableTok{self}\NormalTok{)}
\end{Highlighting}
\end{Shaded}

Creates labels for the actions (for plotting) if none provided

\begin{center}\rule{0.5\linewidth}{\linethickness}\end{center}

\subsubsection{TwoStep.make\_digraph}\label{twostep.make_digraph}

\begin{Shaded}
\begin{Highlighting}[]
\NormalTok{fitr.environments.make_digraph(}\VariableTok{self}\NormalTok{)}
\end{Highlighting}
\end{Shaded}

Creates a \texttt{networkx} \texttt{DiGraph} object from the transition
tensor for the purpose of plotting and some other analyses.

\begin{center}\rule{0.5\linewidth}{\linethickness}\end{center}

\subsubsection{TwoStep.make\_state\_labels}\label{twostep.make_state_labels}

\begin{Shaded}
\begin{Highlighting}[]
\NormalTok{fitr.environments.make_state_labels(}\VariableTok{self}\NormalTok{)}
\end{Highlighting}
\end{Shaded}

Creates labels for the states (for plotting) if none provided

\begin{center}\rule{0.5\linewidth}{\linethickness}\end{center}

\subsubsection{TwoStep.make\_undirected\_graph}\label{twostep.make_undirected_graph}

\begin{Shaded}
\begin{Highlighting}[]
\NormalTok{fitr.environments.make_undirected_graph(}\VariableTok{self}\NormalTok{)}
\end{Highlighting}
\end{Shaded}

Converts the DiGraph to undirected and computes some stats

\begin{center}\rule{0.5\linewidth}{\linethickness}\end{center}

\subsubsection{TwoStep.observation}\label{twostep.observation}

\begin{Shaded}
\begin{Highlighting}[]
\NormalTok{fitr.environments.observation(}\VariableTok{self}\NormalTok{)}
\end{Highlighting}
\end{Shaded}

Samples an initial state from the start-state distribution
\(p(\mathbf x)\)

\[
\mathbf x_0 \sim p(\mathbf x)
\]

Returns:

A one-hot vector \texttt{ndarray((nstates,))} indicating the starting
state.

Examples:

\begin{Shaded}
\begin{Highlighting}[]
\NormalTok{x }\OperatorTok{=}\NormalTok{ env.observation()}
\end{Highlighting}
\end{Shaded}

\begin{center}\rule{0.5\linewidth}{\linethickness}\end{center}

\subsubsection{TwoStep.plot\_action\_outcome\_probabilities}\label{twostep.plot_action_outcome_probabilities}

\begin{Shaded}
\begin{Highlighting}[]
\NormalTok{fitr.environments.plot_action_outcome_probabilities(}\VariableTok{self}\NormalTok{, figsize}\OperatorTok{=}\VariableTok{None}\NormalTok{, outfile}\OperatorTok{=}\VariableTok{None}\NormalTok{, outfiletype}\OperatorTok{=}\StringTok{'pdf'}\NormalTok{, cmap}\OperatorTok{=}\StringTok{'Greys_r'}\NormalTok{)}
\end{Highlighting}
\end{Shaded}

Plots the probabilities of different outcomes given actions.

Each plot is a heatmap for a starting state showing the transition
probabilities for each action-outcome pair within that state.

\begin{center}\rule{0.5\linewidth}{\linethickness}\end{center}

\subsubsection{TwoStep.plot\_graph}\label{twostep.plot_graph}

\begin{Shaded}
\begin{Highlighting}[]
\NormalTok{fitr.environments.plot_graph(}\VariableTok{self}\NormalTok{, figsize}\OperatorTok{=}\VariableTok{None}\NormalTok{, node_size}\OperatorTok{=}\DecValTok{2000}\NormalTok{, arrowsize}\OperatorTok{=}\DecValTok{20}\NormalTok{, lw}\OperatorTok{=}\FloatTok{1.5}\NormalTok{, font_size}\OperatorTok{=}\DecValTok{12}\NormalTok{, title}\OperatorTok{=}\VariableTok{False}\NormalTok{, outfile}\OperatorTok{=}\VariableTok{None}\NormalTok{, outfiletype}\OperatorTok{=}\StringTok{'pdf'}\NormalTok{)}
\end{Highlighting}
\end{Shaded}

Plots the directed graph of the task

\begin{center}\rule{0.5\linewidth}{\linethickness}\end{center}

\subsubsection{TwoStep.plot\_reward\_paths}\label{twostep.plot_reward_paths}

\begin{Shaded}
\begin{Highlighting}[]
\NormalTok{fitr.environments.plot_reward_paths(}\VariableTok{self}\NormalTok{, outfile}\OperatorTok{=}\VariableTok{None}\NormalTok{, outfiletype}\OperatorTok{=}\StringTok{'pdf'}\NormalTok{, figsize}\OperatorTok{=}\VariableTok{None}\NormalTok{)}
\end{Highlighting}
\end{Shaded}

\begin{center}\rule{0.5\linewidth}{\linethickness}\end{center}

\subsubsection{TwoStep.plot\_spectral\_properties}\label{twostep.plot_spectral_properties}

\begin{Shaded}
\begin{Highlighting}[]
\NormalTok{fitr.environments.plot_spectral_properties(}\VariableTok{self}\NormalTok{, figsize}\OperatorTok{=}\VariableTok{None}\NormalTok{, outfile}\OperatorTok{=}\VariableTok{None}\NormalTok{, outfiletype}\OperatorTok{=}\StringTok{'pdf'}\NormalTok{)}
\end{Highlighting}
\end{Shaded}

Creates a set of subplots depicting the graph Laplacian and its spectral
decomposition.

\begin{center}\rule{0.5\linewidth}{\linethickness}\end{center}

\subsubsection{TwoStep.random\_action}\label{twostep.random_action}

\begin{Shaded}
\begin{Highlighting}[]
\NormalTok{fitr.environments.random_action(}\VariableTok{self}\NormalTok{)}
\end{Highlighting}
\end{Shaded}

Samples a random one-hot action vector uniformly over the action space.

Useful for testing that your environment works, without having to create
an agent.

\[
\mathbf u \sim \mathrm{Multinomial}\Big(1, \mathbf p=\{p_i = \frac{1}{|\mathcal U|}\}_{i=1}^{|\mathcal U|}\Big)
\]

Returns:

A one-hot action vector of type \texttt{ndarray((nactions,))}

Examples:

\begin{Shaded}
\begin{Highlighting}[]
\NormalTok{u }\OperatorTok{=}\NormalTok{ env.random_action()}
\end{Highlighting}
\end{Shaded}

\begin{center}\rule{0.5\linewidth}{\linethickness}\end{center}

\subsubsection{TwoStep.step}\label{twostep.step}

\begin{Shaded}
\begin{Highlighting}[]
\NormalTok{fitr.environments.step(}\VariableTok{self}\NormalTok{, action)}
\end{Highlighting}
\end{Shaded}

Executes a state transition in the environment.

Arguments:

action : A one-hot vector of type \texttt{ndarray((naction,))}
indicating the action selected at the current state.

Returns:

A 3-tuple representing the next state (\texttt{ndarray((noutcomes,))}),
scalar reward, and whether the current step terminates a trial
(\texttt{bool}).

Raises:

\texttt{RuntimeError} if \texttt{env.observation()} not called after a
previous \texttt{env.step(...)} call yielded a terminal state.

\begin{center}\rule{0.5\linewidth}{\linethickness}\end{center}

\subsection{ReverseTwoStep}\label{reversetwostep}

\begin{Shaded}
\begin{Highlighting}[]
\NormalTok{fitr.environments.ReverseTwoStep()}
\end{Highlighting}
\end{Shaded}

From Kool \& Gershman 2016.

\begin{center}\rule{0.5\linewidth}{\linethickness}\end{center}

\subsubsection{ReverseTwoStep.adjacency\_matrix\_decomposition}\label{reversetwostep.adjacency_matrix_decomposition}

\begin{Shaded}
\begin{Highlighting}[]
\NormalTok{fitr.environments.adjacency_matrix_decomposition(}\VariableTok{self}\NormalTok{)}
\end{Highlighting}
\end{Shaded}

Singular value decomposition of the graph adjacency matrix

\begin{center}\rule{0.5\linewidth}{\linethickness}\end{center}

\subsubsection{ReverseTwoStep.f\_reward}\label{reversetwostep.f_reward}

\begin{Shaded}
\begin{Highlighting}[]
\NormalTok{fitr.environments.f_reward(}\VariableTok{self}\NormalTok{, R, x)}
\end{Highlighting}
\end{Shaded}

\begin{center}\rule{0.5\linewidth}{\linethickness}\end{center}

\subsubsection{ReverseTwoStep.get\_graph\_depth}\label{reversetwostep.get_graph_depth}

\begin{Shaded}
\begin{Highlighting}[]
\NormalTok{fitr.environments.get_graph_depth(}\VariableTok{self}\NormalTok{)}
\end{Highlighting}
\end{Shaded}

Returns the depth of the task graph.

Calculated as the depth from \texttt{START} (pre-initial state) to
\texttt{END} (which absorbs trial from all terminal states), minus 2 to
account for the \texttt{START-\textgreater{}node} \&
\texttt{node-\textgreater{}END} transitions.

Returns:

An \texttt{int} identifying the depth of the current graph for a single
trial of the task

\begin{center}\rule{0.5\linewidth}{\linethickness}\end{center}

\subsubsection{ReverseTwoStep.laplacian\_matrix\_decomposition}\label{reversetwostep.laplacian_matrix_decomposition}

\begin{Shaded}
\begin{Highlighting}[]
\NormalTok{fitr.environments.laplacian_matrix_decomposition(}\VariableTok{self}\NormalTok{)}
\end{Highlighting}
\end{Shaded}

Singular value decomposition of the graph Laplacian

\begin{center}\rule{0.5\linewidth}{\linethickness}\end{center}

\subsubsection{ReverseTwoStep.make\_action\_labels}\label{reversetwostep.make_action_labels}

\begin{Shaded}
\begin{Highlighting}[]
\NormalTok{fitr.environments.make_action_labels(}\VariableTok{self}\NormalTok{)}
\end{Highlighting}
\end{Shaded}

Creates labels for the actions (for plotting) if none provided

\begin{center}\rule{0.5\linewidth}{\linethickness}\end{center}

\subsubsection{ReverseTwoStep.make\_digraph}\label{reversetwostep.make_digraph}

\begin{Shaded}
\begin{Highlighting}[]
\NormalTok{fitr.environments.make_digraph(}\VariableTok{self}\NormalTok{)}
\end{Highlighting}
\end{Shaded}

Creates a \texttt{networkx} \texttt{DiGraph} object from the transition
tensor for the purpose of plotting and some other analyses.

\begin{center}\rule{0.5\linewidth}{\linethickness}\end{center}

\subsubsection{ReverseTwoStep.make\_state\_labels}\label{reversetwostep.make_state_labels}

\begin{Shaded}
\begin{Highlighting}[]
\NormalTok{fitr.environments.make_state_labels(}\VariableTok{self}\NormalTok{)}
\end{Highlighting}
\end{Shaded}

Creates labels for the states (for plotting) if none provided

\begin{center}\rule{0.5\linewidth}{\linethickness}\end{center}

\subsubsection{ReverseTwoStep.make\_undirected\_graph}\label{reversetwostep.make_undirected_graph}

\begin{Shaded}
\begin{Highlighting}[]
\NormalTok{fitr.environments.make_undirected_graph(}\VariableTok{self}\NormalTok{)}
\end{Highlighting}
\end{Shaded}

Converts the DiGraph to undirected and computes some stats

\begin{center}\rule{0.5\linewidth}{\linethickness}\end{center}

\subsubsection{ReverseTwoStep.observation}\label{reversetwostep.observation}

\begin{Shaded}
\begin{Highlighting}[]
\NormalTok{fitr.environments.observation(}\VariableTok{self}\NormalTok{)}
\end{Highlighting}
\end{Shaded}

Samples an initial state from the start-state distribution
\(p(\mathbf x)\)

\[
\mathbf x_0 \sim p(\mathbf x)
\]

Returns:

A one-hot vector \texttt{ndarray((nstates,))} indicating the starting
state.

Examples:

\begin{Shaded}
\begin{Highlighting}[]
\NormalTok{x }\OperatorTok{=}\NormalTok{ env.observation()}
\end{Highlighting}
\end{Shaded}

\begin{center}\rule{0.5\linewidth}{\linethickness}\end{center}

\subsubsection{ReverseTwoStep.plot\_action\_outcome\_probabilities}\label{reversetwostep.plot_action_outcome_probabilities}

\begin{Shaded}
\begin{Highlighting}[]
\NormalTok{fitr.environments.plot_action_outcome_probabilities(}\VariableTok{self}\NormalTok{, figsize}\OperatorTok{=}\VariableTok{None}\NormalTok{, outfile}\OperatorTok{=}\VariableTok{None}\NormalTok{, outfiletype}\OperatorTok{=}\StringTok{'pdf'}\NormalTok{, cmap}\OperatorTok{=}\StringTok{'Greys_r'}\NormalTok{)}
\end{Highlighting}
\end{Shaded}

Plots the probabilities of different outcomes given actions.

Each plot is a heatmap for a starting state showing the transition
probabilities for each action-outcome pair within that state.

\begin{center}\rule{0.5\linewidth}{\linethickness}\end{center}

\subsubsection{ReverseTwoStep.plot\_graph}\label{reversetwostep.plot_graph}

\begin{Shaded}
\begin{Highlighting}[]
\NormalTok{fitr.environments.plot_graph(}\VariableTok{self}\NormalTok{, figsize}\OperatorTok{=}\VariableTok{None}\NormalTok{, node_size}\OperatorTok{=}\DecValTok{2000}\NormalTok{, arrowsize}\OperatorTok{=}\DecValTok{20}\NormalTok{, lw}\OperatorTok{=}\FloatTok{1.5}\NormalTok{, font_size}\OperatorTok{=}\DecValTok{12}\NormalTok{, title}\OperatorTok{=}\VariableTok{False}\NormalTok{, outfile}\OperatorTok{=}\VariableTok{None}\NormalTok{, outfiletype}\OperatorTok{=}\StringTok{'pdf'}\NormalTok{)}
\end{Highlighting}
\end{Shaded}

Plots the directed graph of the task

\begin{center}\rule{0.5\linewidth}{\linethickness}\end{center}

\subsubsection{ReverseTwoStep.plot\_spectral\_properties}\label{reversetwostep.plot_spectral_properties}

\begin{Shaded}
\begin{Highlighting}[]
\NormalTok{fitr.environments.plot_spectral_properties(}\VariableTok{self}\NormalTok{, figsize}\OperatorTok{=}\VariableTok{None}\NormalTok{, outfile}\OperatorTok{=}\VariableTok{None}\NormalTok{, outfiletype}\OperatorTok{=}\StringTok{'pdf'}\NormalTok{)}
\end{Highlighting}
\end{Shaded}

Creates a set of subplots depicting the graph Laplacian and its spectral
decomposition.

\begin{center}\rule{0.5\linewidth}{\linethickness}\end{center}

\subsubsection{ReverseTwoStep.random\_action}\label{reversetwostep.random_action}

\begin{Shaded}
\begin{Highlighting}[]
\NormalTok{fitr.environments.random_action(}\VariableTok{self}\NormalTok{)}
\end{Highlighting}
\end{Shaded}

Samples a random one-hot action vector uniformly over the action space.

Useful for testing that your environment works, without having to create
an agent.

\[
\mathbf u \sim \mathrm{Multinomial}\Big(1, \mathbf p=\{p_i = \frac{1}{|\mathcal U|}\}_{i=1}^{|\mathcal U|}\Big)
\]

Returns:

A one-hot action vector of type \texttt{ndarray((nactions,))}

Examples:

\begin{Shaded}
\begin{Highlighting}[]
\NormalTok{u }\OperatorTok{=}\NormalTok{ env.random_action()}
\end{Highlighting}
\end{Shaded}

\begin{center}\rule{0.5\linewidth}{\linethickness}\end{center}

\subsubsection{ReverseTwoStep.step}\label{reversetwostep.step}

\begin{Shaded}
\begin{Highlighting}[]
\NormalTok{fitr.environments.step(}\VariableTok{self}\NormalTok{, action)}
\end{Highlighting}
\end{Shaded}

Executes a state transition in the environment.

Arguments:

action : A one-hot vector of type \texttt{ndarray((naction,))}
indicating the action selected at the current state.

Returns:

A 3-tuple representing the next state (\texttt{ndarray((noutcomes,))}),
scalar reward, and whether the current step terminates a trial
(\texttt{bool}).

Raises:

\texttt{RuntimeError} if \texttt{env.observation()} not called after a
previous \texttt{env.step(...)} call yielded a terminal state.

\begin{center}\rule{0.5\linewidth}{\linethickness}\end{center}

\subsection{RandomContextualBandit}\label{randomcontextualbandit}

\begin{Shaded}
\begin{Highlighting}[]
\NormalTok{fitr.environments.RandomContextualBandit()}
\end{Highlighting}
\end{Shaded}

Generates a random bandit task

Arguments:

\begin{itemize}
\tightlist
\item
  \textbf{nactions}: Number of actions
\item
  \textbf{noutcomes}: Number of outcomes
\item
  \textbf{nstates}: Number of contexts
\item
  \textbf{min\_actions\_per\_context}: Different contexts may have more
  or fewer actions than others (never more than \texttt{nactions}). This
  variable describes the minimum number of actions allowed in a context.
\item
  \textbf{alpha}:
\item
  \textbf{alpha\_start}:
\item
  \textbf{shift\_flip}:
\item
  \textbf{reward\_lb}: Lower bound for drifting rewards
\item
  \textbf{reward\_ub}: Upper bound for drifting rewards
\item
  \textbf{reward\_drift}: Values (\texttt{on} or \texttt{off})
  determining whether rewards are allowed to drift
\item
  \textbf{drift\_mu}: Mean of the Gaussian random walk determining
  reward
\item
  \textbf{drift\_sd}: Standard deviation of Gaussian random walk
  determining reward
\end{itemize}

\begin{center}\rule{0.5\linewidth}{\linethickness}\end{center}

\subsubsection{RandomContextualBandit.adjacency\_matrix\_decomposition}\label{randomcontextualbandit.adjacency_matrix_decomposition}

\begin{Shaded}
\begin{Highlighting}[]
\NormalTok{fitr.environments.adjacency_matrix_decomposition(}\VariableTok{self}\NormalTok{)}
\end{Highlighting}
\end{Shaded}

Singular value decomposition of the graph adjacency matrix

\begin{center}\rule{0.5\linewidth}{\linethickness}\end{center}

\subsubsection{RandomContextualBandit.f\_reward}\label{randomcontextualbandit.f_reward}

\begin{Shaded}
\begin{Highlighting}[]
\NormalTok{fitr.environments.f_reward(}\VariableTok{self}\NormalTok{, R, x)}
\end{Highlighting}
\end{Shaded}

\begin{center}\rule{0.5\linewidth}{\linethickness}\end{center}

\subsubsection{RandomContextualBandit.get\_graph\_depth}\label{randomcontextualbandit.get_graph_depth}

\begin{Shaded}
\begin{Highlighting}[]
\NormalTok{fitr.environments.get_graph_depth(}\VariableTok{self}\NormalTok{)}
\end{Highlighting}
\end{Shaded}

Returns the depth of the task graph.

Calculated as the depth from \texttt{START} (pre-initial state) to
\texttt{END} (which absorbs trial from all terminal states), minus 2 to
account for the \texttt{START-\textgreater{}node} \&
\texttt{node-\textgreater{}END} transitions.

Returns:

An \texttt{int} identifying the depth of the current graph for a single
trial of the task

\begin{center}\rule{0.5\linewidth}{\linethickness}\end{center}

\subsubsection{RandomContextualBandit.laplacian\_matrix\_decomposition}\label{randomcontextualbandit.laplacian_matrix_decomposition}

\begin{Shaded}
\begin{Highlighting}[]
\NormalTok{fitr.environments.laplacian_matrix_decomposition(}\VariableTok{self}\NormalTok{)}
\end{Highlighting}
\end{Shaded}

Singular value decomposition of the graph Laplacian

\begin{center}\rule{0.5\linewidth}{\linethickness}\end{center}

\subsubsection{RandomContextualBandit.make\_action\_labels}\label{randomcontextualbandit.make_action_labels}

\begin{Shaded}
\begin{Highlighting}[]
\NormalTok{fitr.environments.make_action_labels(}\VariableTok{self}\NormalTok{)}
\end{Highlighting}
\end{Shaded}

Creates labels for the actions (for plotting) if none provided

\begin{center}\rule{0.5\linewidth}{\linethickness}\end{center}

\subsubsection{RandomContextualBandit.make\_digraph}\label{randomcontextualbandit.make_digraph}

\begin{Shaded}
\begin{Highlighting}[]
\NormalTok{fitr.environments.make_digraph(}\VariableTok{self}\NormalTok{)}
\end{Highlighting}
\end{Shaded}

Creates a \texttt{networkx} \texttt{DiGraph} object from the transition
tensor for the purpose of plotting and some other analyses.

\begin{center}\rule{0.5\linewidth}{\linethickness}\end{center}

\subsubsection{RandomContextualBandit.make\_state\_labels}\label{randomcontextualbandit.make_state_labels}

\begin{Shaded}
\begin{Highlighting}[]
\NormalTok{fitr.environments.make_state_labels(}\VariableTok{self}\NormalTok{)}
\end{Highlighting}
\end{Shaded}

Creates labels for the states (for plotting) if none provided

\begin{center}\rule{0.5\linewidth}{\linethickness}\end{center}

\subsubsection{RandomContextualBandit.make\_undirected\_graph}\label{randomcontextualbandit.make_undirected_graph}

\begin{Shaded}
\begin{Highlighting}[]
\NormalTok{fitr.environments.make_undirected_graph(}\VariableTok{self}\NormalTok{)}
\end{Highlighting}
\end{Shaded}

Converts the DiGraph to undirected and computes some stats

\begin{center}\rule{0.5\linewidth}{\linethickness}\end{center}

\subsubsection{RandomContextualBandit.observation}\label{randomcontextualbandit.observation}

\begin{Shaded}
\begin{Highlighting}[]
\NormalTok{fitr.environments.observation(}\VariableTok{self}\NormalTok{)}
\end{Highlighting}
\end{Shaded}

Samples an initial state from the start-state distribution
\(p(\mathbf x)\)

\[
\mathbf x_0 \sim p(\mathbf x)
\]

Returns:

A one-hot vector \texttt{ndarray((nstates,))} indicating the starting
state.

Examples:

\begin{Shaded}
\begin{Highlighting}[]
\NormalTok{x }\OperatorTok{=}\NormalTok{ env.observation()}
\end{Highlighting}
\end{Shaded}

\begin{center}\rule{0.5\linewidth}{\linethickness}\end{center}

\subsubsection{RandomContextualBandit.plot\_action\_outcome\_probabilities}\label{randomcontextualbandit.plot_action_outcome_probabilities}

\begin{Shaded}
\begin{Highlighting}[]
\NormalTok{fitr.environments.plot_action_outcome_probabilities(}\VariableTok{self}\NormalTok{, figsize}\OperatorTok{=}\VariableTok{None}\NormalTok{, outfile}\OperatorTok{=}\VariableTok{None}\NormalTok{, outfiletype}\OperatorTok{=}\StringTok{'pdf'}\NormalTok{, cmap}\OperatorTok{=}\StringTok{'Greys_r'}\NormalTok{)}
\end{Highlighting}
\end{Shaded}

Plots the probabilities of different outcomes given actions.

Each plot is a heatmap for a starting state showing the transition
probabilities for each action-outcome pair within that state.

\begin{center}\rule{0.5\linewidth}{\linethickness}\end{center}

\subsubsection{RandomContextualBandit.plot\_graph}\label{randomcontextualbandit.plot_graph}

\begin{Shaded}
\begin{Highlighting}[]
\NormalTok{fitr.environments.plot_graph(}\VariableTok{self}\NormalTok{, figsize}\OperatorTok{=}\VariableTok{None}\NormalTok{, node_size}\OperatorTok{=}\DecValTok{2000}\NormalTok{, arrowsize}\OperatorTok{=}\DecValTok{20}\NormalTok{, lw}\OperatorTok{=}\FloatTok{1.5}\NormalTok{, font_size}\OperatorTok{=}\DecValTok{12}\NormalTok{, title}\OperatorTok{=}\VariableTok{False}\NormalTok{, outfile}\OperatorTok{=}\VariableTok{None}\NormalTok{, outfiletype}\OperatorTok{=}\StringTok{'pdf'}\NormalTok{)}
\end{Highlighting}
\end{Shaded}

Plots the directed graph of the task

\begin{center}\rule{0.5\linewidth}{\linethickness}\end{center}

\subsubsection{RandomContextualBandit.plot\_spectral\_properties}\label{randomcontextualbandit.plot_spectral_properties}

\begin{Shaded}
\begin{Highlighting}[]
\NormalTok{fitr.environments.plot_spectral_properties(}\VariableTok{self}\NormalTok{, figsize}\OperatorTok{=}\VariableTok{None}\NormalTok{, outfile}\OperatorTok{=}\VariableTok{None}\NormalTok{, outfiletype}\OperatorTok{=}\StringTok{'pdf'}\NormalTok{)}
\end{Highlighting}
\end{Shaded}

Creates a set of subplots depicting the graph Laplacian and its spectral
decomposition.

\begin{center}\rule{0.5\linewidth}{\linethickness}\end{center}

\subsubsection{RandomContextualBandit.random\_action}\label{randomcontextualbandit.random_action}

\begin{Shaded}
\begin{Highlighting}[]
\NormalTok{fitr.environments.random_action(}\VariableTok{self}\NormalTok{)}
\end{Highlighting}
\end{Shaded}

Samples a random one-hot action vector uniformly over the action space.

Useful for testing that your environment works, without having to create
an agent.

\[
\mathbf u \sim \mathrm{Multinomial}\Big(1, \mathbf p=\{p_i = \frac{1}{|\mathcal U|}\}_{i=1}^{|\mathcal U|}\Big)
\]

Returns:

A one-hot action vector of type \texttt{ndarray((nactions,))}

Examples:

\begin{Shaded}
\begin{Highlighting}[]
\NormalTok{u }\OperatorTok{=}\NormalTok{ env.random_action()}
\end{Highlighting}
\end{Shaded}

\begin{center}\rule{0.5\linewidth}{\linethickness}\end{center}

\subsubsection{RandomContextualBandit.step}\label{randomcontextualbandit.step}

\begin{Shaded}
\begin{Highlighting}[]
\NormalTok{fitr.environments.step(}\VariableTok{self}\NormalTok{, action)}
\end{Highlighting}
\end{Shaded}

Executes a state transition in the environment.

Arguments:

action : A one-hot vector of type \texttt{ndarray((naction,))}
indicating the action selected at the current state.

Returns:

A 3-tuple representing the next state (\texttt{ndarray((noutcomes,))}),
scalar reward, and whether the current step terminates a trial
(\texttt{bool}).

Raises:

\texttt{RuntimeError} if \texttt{env.observation()} not called after a
previous \texttt{env.step(...)} call yielded a terminal state.

\begin{center}\rule{0.5\linewidth}{\linethickness}\end{center}
