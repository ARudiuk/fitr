\section{\texorpdfstring{\texttt{fitr.data}}{fitr.data}}\label{fitr.data}

A module containing a generic class for behavioural data.

\subsection{BehaviouralData}\label{behaviouraldata}

\begin{Shaded}
\begin{Highlighting}[]
\NormalTok{fitr.data.BehaviouralData()}
\end{Highlighting}
\end{Shaded}

A flexible and generic object to store and process behavioural data
across tasks

Arguments:

\begin{itemize}
\tightlist
\item
  \textbf{ngroups}: Integer number of groups represented in the dataset.
  Only \textgreater{} 1 if data are merged
\item
  \textbf{nsubjects}: Integer number of subjects in dataset
\item
  \textbf{ntrials}: Integer number of trials done by each subject
\item
  \textbf{dict}: Dictionary storage indexed by subject.
\item
  \textbf{params}: \texttt{ndarray((nsubjects,\ nparams\ +\ 1))}
  parameters for each (simulated) subject
\item
  \textbf{meta}: Array of covariates of type
  \texttt{ndarray((nsubjects,\ nmetadata\_features+1))}
\item
  \textbf{tensor}: Tensor representation of the behavioural data of type
  \texttt{ndarray((nsubjects,\ ntrials,\ nfeatures))}
\end{itemize}

\begin{center}\rule{0.5\linewidth}{\linethickness}\end{center}

\subsubsection{BehaviouralData.add\_subject}\label{behaviouraldata.add_subject}

\begin{Shaded}
\begin{Highlighting}[]
\NormalTok{fitr.data.add_subject(}\VariableTok{self}\NormalTok{, subject_index, parameters, subject_meta)}
\end{Highlighting}
\end{Shaded}

Appends a new subject to the dataset

Arguments:

\begin{itemize}
\tightlist
\item
  \textbf{subject\_index}: Integer identification for subject
\item
  \textbf{parameters}: \texttt{list} of parameters for the subject
\item
  \textbf{subject\_meta}: Some covariates for the subject
  (\texttt{list})
\end{itemize}

\begin{center}\rule{0.5\linewidth}{\linethickness}\end{center}

\subsubsection{BehaviouralData.initialize\_data\_dictionary}\label{behaviouraldata.initialize_data_dictionary}

\begin{Shaded}
\begin{Highlighting}[]
\NormalTok{fitr.data.initialize_data_dictionary(}\VariableTok{self}\NormalTok{)}
\end{Highlighting}
\end{Shaded}

\begin{center}\rule{0.5\linewidth}{\linethickness}\end{center}

\subsubsection{BehaviouralData.make\_behavioural\_ngrams}\label{behaviouraldata.make_behavioural_ngrams}

\begin{Shaded}
\begin{Highlighting}[]
\NormalTok{fitr.data.make_behavioural_ngrams(}\VariableTok{self}\NormalTok{, n)}
\end{Highlighting}
\end{Shaded}

Creates N-grams of behavioural data

\begin{center}\rule{0.5\linewidth}{\linethickness}\end{center}

\subsubsection{BehaviouralData.make\_cooccurrence\_matrix}\label{behaviouraldata.make_cooccurrence_matrix}

\begin{Shaded}
\begin{Highlighting}[]
\NormalTok{fitr.data.make_cooccurrence_matrix(}\VariableTok{self}\NormalTok{, k, dtype}\OperatorTok{=<}\KeywordTok{class} \StringTok{'numpy.float32'}\OperatorTok{>}\NormalTok{)}
\end{Highlighting}
\end{Shaded}

\begin{center}\rule{0.5\linewidth}{\linethickness}\end{center}

\subsubsection{BehaviouralData.make\_tensor\_representations}\label{behaviouraldata.make_tensor_representations}

\begin{Shaded}
\begin{Highlighting}[]
\NormalTok{fitr.data.make_tensor_representations(}\VariableTok{self}\NormalTok{)}
\end{Highlighting}
\end{Shaded}

Creates a tensor with all subjects' data

\paragraph{Notes}\label{notes}

Assumes that all subjects did same number of trials.

\begin{center}\rule{0.5\linewidth}{\linethickness}\end{center}

\subsubsection{BehaviouralData.numpy\_tensor\_to\_bdf}\label{behaviouraldata.numpy_tensor_to_bdf}

\begin{Shaded}
\begin{Highlighting}[]
\NormalTok{fitr.data.numpy_tensor_to_bdf(}\VariableTok{self}\NormalTok{, X)}
\end{Highlighting}
\end{Shaded}

Creates \texttt{BehaviouralData} formatted set from a dataset stored in
a numpy \texttt{ndarray}.

Arguments:

\begin{itemize}
\tightlist
\item
  \textbf{X}: \texttt{ndarray((nsubjects,\ ntrials,\ m))} with
  \texttt{m} being the size of flattened single-trial data
\end{itemize}

\begin{center}\rule{0.5\linewidth}{\linethickness}\end{center}

\subsubsection{BehaviouralData.unpack\_tensor}\label{behaviouraldata.unpack_tensor}

\begin{Shaded}
\begin{Highlighting}[]
\NormalTok{fitr.data.unpack_tensor(}\VariableTok{self}\NormalTok{, x_dim, u_dim, r_dim}\OperatorTok{=}\DecValTok{1}\NormalTok{, terminal_dim}\OperatorTok{=}\DecValTok{1}\NormalTok{, get}\OperatorTok{=}\StringTok{'sarsat'}\NormalTok{)}
\end{Highlighting}
\end{Shaded}

Unpacks data stored in tensor format into separate arrays for states,
actions, rewards, next states, and next actions.

Arguments:

x\_dim : Task state space dimensionality (\texttt{int}) u\_dim : Task
action space dimensionality (\texttt{int}) r\_dim : Reward
dimensionality (\texttt{int}, default=1) terminal\_dim : Dimensionality
of the terminal state indicator (\texttt{int} , default=1) get : String
indicating the order that data are stored in the array. Can also be
shortened such that fewer elements are returned. For example, the
default is \texttt{sarsat}.

Returns:

List with data, where each element is in the order of the argument
\texttt{get}

\begin{center}\rule{0.5\linewidth}{\linethickness}\end{center}

\subsubsection{BehaviouralData.update}\label{behaviouraldata.update}

\begin{Shaded}
\begin{Highlighting}[]
\NormalTok{fitr.data.update(}\VariableTok{self}\NormalTok{, subject_index, behav_data)}
\end{Highlighting}
\end{Shaded}

Adds behavioural data to the dataset

Arguments:

\begin{itemize}
\tightlist
\item
  \textbf{subject\_index}: Integer index for the subject
\item
  \textbf{behav\_data}: 1-dimensional \texttt{ndarray} of flattened data
\end{itemize}

\begin{center}\rule{0.5\linewidth}{\linethickness}\end{center}

\subsection{merge\_behavioural\_data}\label{merge_behavioural_data}

\begin{Shaded}
\begin{Highlighting}[]
\NormalTok{fitr.data.merge_behavioural_data(datalist)}
\end{Highlighting}
\end{Shaded}

Combines BehaviouralData objects.

Arguments:

\begin{itemize}
\tightlist
\item
  \textbf{datalist}: List of BehaviouralData objects
\end{itemize}

Returns:

\texttt{BehaviouralData} with data from multiple groups merged.

\begin{center}\rule{0.5\linewidth}{\linethickness}\end{center}
